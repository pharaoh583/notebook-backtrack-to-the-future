\documentclass[10pt, landscape, twocolumn, a4paper, notitlepage]{article}
\usepackage{xeCJK}
\setCJKmainfont{IPAMincho}
\setCJKsansfont{IPAGothic}
\setCJKmonofont{IPAGothic}
\usepackage{hyperref}
\usepackage[spanish, activeacute]{babel}
\usepackage{fancyhdr}
\usepackage{lastpage}
\usepackage{listings}
\usepackage{amssymb}
\usepackage{svnkw}
\usepackage{rotating}
\usepackage{amsmath}

%%% Márgenes
\setlength{\columnsep}{0.25in}    % default=10pt
\setlength{\columnseprule}{0.5pt}    % default=0pt (no line)

\addtolength{\textheight}{2.35in}
\addtolength{\topmargin}{-0.9in}     % ~ -0.5 del incremento anterior

\addtolength{\textwidth}{1.1in}
\addtolength{\oddsidemargin}{-0.60in} % -0.5 del incremento anterior
\setlength{\headwidth}{\textwidth}
\addtolength{\headwidth}{0.1in}

\setlength{\headsep}{0.08in}
\setlength{\parskip}{0in}
\setlength{\headheight}{15pt}
\setlength{\parindent}{0mm}

%%% Encabezado y pie de página
\pagestyle{fancy}
\fancyhead[LO]{\leftmark\ -\ \rightmark}
%\fancyhead[C]{\textbf{AJI-UBA}}
\fancyhead[RO]{\textbf{(FFT) Final Fourier Tactics} - rev \revision\ - P\'agina \thepage\ de \pageref{LastPage}\ \begin{rotate}{270}\hspace{.8em}\underline{\textbf{Universidad Cat\'olica Boliviana - UCB }\hspace{4.1in}\ }\end{rotate}}
\fancyfoot{}
\fancyfoot[RO]{\begin{rotate}{270}\hspace{-10em}\textbf{Page \thepage\ of \pageref{LastPage}}\end{rotate}}
\renewcommand{\headrulewidth}{0.4pt}
\renewcommand{\footrulewidth}{0.4pt}
\renewcommand{\footruleskip}{0.2in}

%%% Configuración de Listings
\lstloadlanguages{C++}
\lstnewenvironment{code}
	{%\lstset{	numbers=none, frame=lines, basicstyle=\small\ttfamily, }%
	 \csname lst@SetFirstLabel\endcsname}
	{\csname lst@SaveFirstLabel\endcsname}
\lstset{% general command to set parameter(s)
	language=C++, basicstyle=\small\ttfamily, keywordstyle=\slshape,
	emph=[1]{tipo,usa}, emphstyle={[1]\sffamily\bfseries},
	morekeywords={tint,forn,forsn},
	basewidth={0.47em,0.40em},
	columns=fixed, fontadjust, resetmargins, xrightmargin=5pt, xleftmargin=15pt,
	flexiblecolumns=false, tabsize=2, breaklines,	breakatwhitespace=false, extendedchars=true,
	numbers=left, numberstyle=\tiny, stepnumber=1, numbersep=9pt,
	frame=l, framesep=3pt,
}

\newcommand{\comb}[2]{{}_{#1}\mathrm{C}_{#2}}
\newcommand{\threepartdef}[6]
{
	\left\{
		\begin{array}{lll}
			#1 & \mbox{if } #2 \\
			#3 & \mbox{if } #4 \\
			#5 & \mbox{if } #6
		\end{array}
	\right.
}

\begin{document}

\tableofcontents\newpage

%%% El texto propiamente dicho
\nbtitle{(FFT) Final Fourier Tactics - Reference}
\section{Algoritmos}
Para los randoms agregar al inicio del main
\begin{code}
srand ( unsigned ( std::time(0) ) ); //cambiar la semilla si WA
double r = ((double) rand() / (RAND_MAX)); //0 <= r <= 1 
\end{code}

\section{Tips}
For problem solving
\begin{itemize}
  \item What if the problem can be intepreted as a graph, maybe is a state space search with BFS/Dijkstra/meet in the middle
  \item Try binary (bits, binary search) 
  \item Try to find a recurrence relation
  \item Try the first idea that works and try to optimize later, maybe with a ds
  \item If the complexity is tight implement and check if fits
  \item Don't think same things over and over
  \item Try sqrt decomposition 
  \item If some thing can only be used/activated a limited number of times try flows
  \item If you’re stuck: Try small cases, simulate and print
  \item Think from both ends (two pointers, meet-in-the-middle, or bidirectional BFS).
  \item Try reducing the problem to a known one, specially when dealing with np problems
\end{itemize}

For debugging
\begin{itemize}
    \item If there are multiple test cases add two consecutive test that are equal
    \item Check if you are restarting values (specially on graph problems with global values)
    \item Try a brute-force or slow version to compare outputs 
\end{itemize}

In case of Time Limit
\begin{itemize}
    \item Use bitwise tricks/bitsets if possible $(x & (x - 1)) == 0$ to check power of two.
    \item Use array instead of vector
    \item int operations are faster than long long ones
    \item Use iterative instead of recursive
    \item Use only required libraries intead of bits/stdc++.h
Before reducing the constant of your algorithm check if you can reduce complexity by using better algorithms/data structures
for example:
$O(2^n) > O(2^{n−1})$
\end{itemize}

\section{Template}
\begin{code}
#include <bits/stdc++.h>
using namespace std;
#define forn(i, n) for(int i=0;i<int(n);i++)
#define forsn(i,s,n) for(int i=int(s);i<int(n);i++)
#define all(v) (v).begin(),(v).end()
#define FastIO ios_base::sync_with_stdio(false);cin.tie(NULL)
#define ll long long
#define ii pair<int,int>
#define vi vector<int>
#define F first
#define S second
#define pb push_back
#define mp make_pair
#define el "\n"
#define dbg(x) cout<<#x<<" = "<<(x)<<el;
void solve(){
    //TODO: implement solution here
}
int main(){
    FastIO;
    int t = 1; 
    //int t; cin>>t;
    while(t--) solve();
}
\end{code}
\newpage

\section{Estructuras}
\subsection{Mo's}
\begin{code}
inline int64_t hilbertOrder(int x, int y, int pow, int rotate) {
    if (pow == 0) return 0;
    int hpow = 1 << (pow-1);
    int seg = (x < hpow) 
            ? ((y < hpow) ? 0 : 3) 
            : ((y < hpow) ? 1 : 2);
    seg = (seg + rotate) & 3;
    const int rotateDelta[4] = {3, 0, 0, 1};
    int nx = x & (x ^ hpow), ny = y & (y ^ hpow);
    int nrot = (rotate + rotateDelta[seg]) & 3;
    int64_t subSquareSize = int64_t(1) << (2*pow - 2);
    int64_t ans = seg * subSquareSize;
    int64_t add = hilbertOrder(nx, ny, pow-1, nrot);
    ans += (seg == 1 || seg == 2) ? add : (subSquareSize - add - 1);
    return ans;
}
const int bs = 450;
struct Query {
    int l, r, idx;
    int64_t ord;
    inline void calcOrder() {
        ord = hilbertOrder(l, r, 21, 0);
    }
    bool operator <(query &other) const {
        return ord < other.ord;
    }
    static bool cmp(pair<int, int> p, pair<int, int> q) {
        if (p.first / bs != q.first / bs)
            return p < q;
        return (p.first / bs & 1) ? (p.second < q.second) : (p.second > q.second);
    }
};
//TODO: implement add function add(idx)
//TODO: implement remove function remove(idx)
//TODO: find way to get answer
vector<int> mo_s_algorithm(vector<Query> queries) {
    vector<int> answers(queries.size());
    sort(all(queries);
    // TODO: initialize data structure
    int cur_l = 0;
    int cur_r = -1;
    for (Query q : queries) {
        while (cur_l > q.l) {
            cur_l--;add(cur_l);
        }
        while (cur_r < q.r) {
            cur_r++;add(cur_r);
        }
        while (cur_l < q.l) {
            remove(cur_l);cur_l++;
        }
        while (cur_r > q.r) {
            remove(cur_r);cur_r--;
        }
        answers[q.idx] = get_answer();
    }
    return answers;
}
\end{code}
\subsection{Treap implicit}
Example that supports range reverse and addition updates, and range sum query
(commented parts are specific to this  problem)
\begin{code}
typedef struct item *pitem;
struct item {
	int pr,cnt,val;
//	int sum; // (paramters for range query)
//	bool rev;int add; // (parameters for lazy prop)
	pitem l,r;
	item(int val): pr(rand()),cnt(1),val(val),l(0),r(0)/*,sum(val),rev(0),add(0)*/ {}
};
void push(pitem it){
	if(it){
		/*if(it->rev){
			swap(it->l,it->r);
			if(it->l)it->l->rev^=true;
			if(it->r)it->r->rev^=true;
			it->rev=false;
		}
		it->val+=it->add;it->sum+=it->cnt*it->add;
		if(it->l)it->l->add+=it->add;
		if(it->r)it->r->add+=it->add;
		it->add=0;*/
	}
}
int cnt(pitem t){return t?t->cnt:0;}
// int sum(pitem t){return t?push(t),t->sum:0;}
void upd_cnt(pitem t){
	if(t){
		t->cnt=cnt(t->l)+cnt(t->r)+1;
		// t->sum=t->val+sum(t->l)+sum(t->r);
	}
}
void merge(pitem& t, pitem l, pitem r){
	push(l);push(r);
	if(!l||!r)t=l?l:r;
	else if(l->pr>r->pr)merge(l->r,l->r,r),t=l;
	else merge(r->l,l,r->l),t=r;
	upd_cnt(t);
}
void split(pitem t, pitem& l, pitem& r, int sz){ // sz:desired size of l
	if(!t){l=r=0;return;}
	push(t);
	if(sz<=cnt(t->l))split(t->l,l,t->l,sz),r=t;
	else split(t->r,t->r,r,sz-1-cnt(t->l)),l=t;
	upd_cnt(t);
}
void output(pitem t){ // useful for debugging
	if(!t)return;
	push(t);
	output(t->l);printf(" %d",t->val);output(t->r);
}
// use merge and split for range updates and queries
\end{code}
\subsection{Heavy-Light Decomposition}
\begin{code}
//TODO: implement oper function
//TODO: define MAXN
//TODO: define NEUT neutral value for queries on SegTree and hld
//TODO: implement SegTree, queries should be [l, r)
vi g[MAXN];
int wg[MAXN],dad[MAXN],dep[MAXN]; // weight,father,depth
void dfs1(int x){
	wg[x]=1;
	for(int y:g[x])if(y!=dad[x]){
		dad[y]=x;dep[y]=dep[x]+1;dfs1(y);
		wg[x]+=wg[y];
	}
}
int curpos,pos[MAXN],head[MAXN];
void hld(int x, int c){
	if(c<0)c=x;
	pos[x]=curpos++;head[x]=c;
	int mx=-1;
	for(int y:g[x])if(y!=dad[x]&&(mx<0||wg[mx]<wg[y]))mx=y;
	if(mx>=0)hld(mx,c);
	for(int y:g[x])if(y!=mx&&y!=dad[x])hld(y,-1);
}
void hld_init(){dad[0]=-1;dep[0]=0;dfs1(0);curpos=0;hld(0,-1);}
ll query(int x, int y, SegTree& rmq){
	ll r=NEUT;
	while(head[x]!=head[y]){
		if(dep[head[x]]>dep[head[y]])swap(x,y);
		r=oper(r,rmq.query(pos[head[y]],pos[y]+1));
		y=dad[head[y]];
	}
	if(dep[x]>dep[y])swap(x,y); // now x is lca
	r=oper(r,rmq.query(pos[x],pos[y]+1));
	return r;
}
// for creation call hld_init after reading g then create z such that z[pos[i]] = cost[i]
// init SegTree with z, then you can start doing queries
// for updating: rmq.upd(pos[x],v);
// queries on edges: - assign values of edges to "child" node
//                   - change pos[x] to pos[x]+1 i
\end{code}
\subsection{Centroid decomposition}
\begin{code}
vi s;
vi g[200001];
int n;
vector<bool> disabled;
int treeSize(int u, int p) {
    int sz = 1;
    for(auto v : g[u]) {
        if(v != p and !disabled[v]) {
            sz += treeSize(v, u);
        }
    }
    return s[u] = sz;
} 
int findCentroid(int u, int size) {
    for(auto v : g[u]) {
        if(disabled[v]) continue;
        if(s[v]*2 > size) {
            s[u] -= s[v];  
            return findCentroid(v, size);
        }
    }
    return u;
}
ll centroidDecomposition(int provRoot = 0) {
    int sz = treeSize(provRoot, provRoot);
    //TODO: check if you can end early based on size
    int centroid = findCentroid(provRoot, sz);
    ll res = 0;
    //TODO: implement logic to process centroid
    disabled[centroid] = true;
    //TODO: clear data if necessary
    return res;
}
//TODO: start disabled and s
\end{code}

\subsection{El versatil}
\begin{code}
#define NEUT INF
typedef int T;
T op(T a, T b) {return min(a, b);}
struct STree {
    vector<T> st; int n;
    STree(int n) : st(4 * n + 5, NEUT), n(n) {}
    void init(int p, int l, int r, vector<T>& a) {
        if(l == r) {st[p]=a[l]; return;}
        int m = (l+r)/2;
        init(p<<1, l, m, a); init(p<<1|1, m + 1, r, a);
        st[p] = oper(st[p<<1], st[p<<1|1]);
    }
    void upd(int p, int l, int r, int pos, T val) {
        if(l == r) {st[p] = val; return;}
        int m = (l + r)/2;
        if(pos <= m) upd(p<<1, l, m, pos, val);
        else upd(p<<1|1, m + 1, r, pos, val);
        st[p] = oper(st[p<<1], st[p<<1|1]);
    }
    T query(int p, int l, int r, int a, int b) { //query of [l, r]
        if(l > b or r < a) return NEUT;
        if(a <= l and r <= b) return st[p];
        int m = (l + r)/2;
        return oper(query(p<<1, l, m, a, b), query(p<<1|1, m + 1, r, a, b));
    }

    void init(vector<T>& a){init(1, 0, n - 1, a);}
    void upd(int p, T val) {upd(1, 0, n - 1, p, v);}
    T query(int a, int b) {return query(1, 0, n - 1, a, b);}
};
\end{code}
\subsection{El versatil 2D}
\begin{code}
int n, m; 
const int MAXN = 1000;
int a[MAXN][MAXN], st[2*MAXN][2*MAXN];
#define NEUT 0
int oper(int a, int b) {return a + b;}
void build() {
    forn(i, n) forn(j, m) st[i + n][j + m] = a[i][j];
    forn(i, n) for(int j = m - 1; j; --j)
        st[i + n][j] = oper(st[i + n][j<<1], st[i + n][j<<1|1]);
    for(int i = n - 1; i; --i) forn(j, m<<1) {
        st[i][j] = oper(st[i<<1][j], st[i<<1|1][j]);
    }
}
void upd(int r, int c, int v){
    st[r+n][c+m] = v;
    for(int j=c+m; j>1; j>>=1) st[r+n][j>>1]=oper(st[r+n][j], st[r+n][j^1]);
    for(int i = r+n; i>1; i>>=1) for(int j = c+m; j; j>>=1)
        st[i>>1][j]=oper(st[i][j], st[i^1][j]);
}
//query of [r0, r1), [c0, c1)
int query(int r0, int r1, int c0, int c1) {
    int r = NEUT;
    for(int i0=r0+n, i1=r1+n; i0<i1; i0>>=1, i1>>=1) {
        int t[4], q = 0;
        if(i0&1) t[q++]=i0++;
        if(i1&1) t[q++]=--i1;
        forn(k, q) for(int j0=c0+m, j1=c1+m; j0<j1; j0>>=1, j1>>=1) {
            if(j0&1) r=oper(r, st[t[k]][j0++]);
            if(j1&1) r=oper(r, st[t[k]][--j1]);
        }
    }
    return r;
}
\end{code}
\subsection{Sparse Table}
\begin{code}
typedef int T;
T op(T a, T b) {return min(a, b);}
struct Table{
    vector<vector<T>> st;
    Table (vector<T>& v) : st(1, v) {
        int n = v.size();
        for(int j = 1; (1<<j) <= n; j++) {
            st.emplace_back(vector<T>(n));
            for(int i = 0; i + (1<<(j - 1)) < n; i++) {
                st[j][i] = op(st[j - 1][i], st[j - 1][i + (1<<(j - 1))]);
            }
        }
    }
    T get(int l, int r) { // get [l, r]
        int j = 31-__builtin_clz(r - l + 1);
        return op(st[j][l], st[j][r - (1<<j) + 1]);
    }
};
\end{code}
\section{Secci\'on de Andres}
\subsection{Mobius}
Count numbers coprime to n; $\varphi(n)$\\
if n is prime then $\varphi(n) = n - 1$\\
if n is a power of a prime then $\varphi(p^k) = p^k - p^(k-1)$\\
\\
Count divisors of n; $\sigma(n)$\\
if n is a prime to the power of k $\sigma(n) = k + 1$\\
\\
Mobius function is defined as: \\
$\mu(n) = \threepartdef
{1}      {n=1}
{0}      {a^2 | n \mbox{ for some } a > 1}
{(-1)^r} {n \mbox{ has } r \mbox{ distinct prime factors}}$\\
\\
These three are multiplicative functions\\
Then, the mobius invertion is:\\
if $g(n) = \sum_{d|n} f(d)$ \\
then $f(n) = \sum_{d|n} \mu(d)g(\frac{n}{d})$

\begin{code}
//TODO: add multiplicative functions
//TODO: define MAXN
short mu[MAXN] = {0,1};
void mobius(){
	forsn(i,1,MAXN)if(mu[i])for(int j=i+i;j<MAXN;j+=i)mu[j]-=mu[i];
}
\end{code}
\subsection{Chinese Remainder Theorem}
\begin{code}
ll gcd(ll a, ll b){while(b){ll t=a%b;a=b;b=t;}return a;}
pair<ll,ll> extendedEuclid (ll a, ll b){ //a * x + b * y = gcd(a,b)
	ll x,y;
	if (b==0) return {1,0};
	auto p=extendedEuclid(b,a%b);
	x=p.snd;
	y=p.fst-(a/b)*x;
	if(a*x+b*y==-gcd(a,b)) x=-x, y=-y;
	return {x,y};
}
pair<pair<ll,ll>,pair<ll,ll> > diophantine(ll a,ll b, ll r) {
	//a*x+b*y=r where r is multiple of gcd(a,b);
	ll d=gcd(a,b);
	a/=d; b/=d; r/=d;
	auto p = extendedEuclid(a,b);
	p.fst*=r; p.snd*=r;
	assert(a*p.fst+b*p.snd==r);
	return {p,{-b,a}}; // solutions: p+t*ans.snd
}
ll inv(ll a, ll m) {
	assert(gcd(a,m)==1);
	ll x = diophantine(a,m,1).fst.fst;
	return ((x%m)+m)%m;
}
#define mod(a,m) (((a)%m+m)%m)
pair<ll,ll> sol(tuple<ll,ll,ll> c){ //requires inv, diophantine
    ll a=get<0>(c), x1=get<1>(c), m=get<2>(c), d=gcd(a,m);
    if(d==1) return {mod(x1*inv(a,m),m), m};
    else return x1%d ? ii({-1LL,-1LL}) : sol(make_tuple(a/d,x1/d,m/d));
}
pair<ll,ll> crt(vector< tuple<ll,ll,ll> > cond) { // returns: (sol, lcm)
	ll x1=0,m1=1,x2,m2;
	for(auto t:cond){
		tie(x2,m2)=sol(t);
		if((x1-x2)%gcd(m1,m2))return {-1,-1};
		if(m1==m2)continue;
		ll k=diophantine(m2,-m1,x1-x2).fst.snd,l=m1*(m2/gcd(m1,m2));
		x1=mod((__int128)m1*k+x1,l);m1=l;
	}
	return sol(make_tuple(1,x1,m1));
} //cond[i]={ai,bi,mi} ai*xi=bi (mi); assumes lcm fits in ll
\end{code}
\subsection{Discrete log}
Returns x such that $a^x = b (mod m)$ or -1 if inexistent in $O(\sqrt{m})$
\begin{code}
ll discrete_log(ll a,ll b,ll m) {
    a%=m, b%=m;
    if(b == 1) return 0;
    int cnt=0;
    ll tmp=1;
    for(int g=__gcd(a,m);g!=1;g=__gcd(a,m)) {
        if(b%g) return -1;
        m/=g, b/=g;
        tmp = tmp*a/g%m;
        ++cnt;
        if(b == tmp) return cnt;
    }
    map<ll,int> w;
    int s = ceil(sqrt(m));
    ll base = b;
    forn(i,s) {
        w[base] = i;
        base=base*a%m;
    }
    base=fastpow(a,s,m);
    ll key=tmp;
    forsn(i,1,s+2) {
        key=base*key%m;
        if(w.count(key)) return i*s-w[key]+cnt;
    }
    return -1;
}
\end{code}
\subsection{Factoriales e Inversas}
\begin{code}
const ll M = 998244353; //TODO: change mod
int maxi = 200010;
ll F[200010], INV[200010], FI[200010];
void init(){
	F[0] = 1; forsn(i, 1, maxi) F[i] = F[i-1]*i %M;
	INV[1] = 1; forsn(i, 2, maxi) INV[i] = M - (ll)(M/i)*INV[M%i]%M;
	FI[0] = 1; forsn(i, 1, maxi) FI[i] = FI[i-1]*INV[i] %M;
}
ll modInverse(ll a){
    return modPower(a, M - 2); //TODO: implement binary exp
}
\end{code}
\subsection{Teoremas y propiedades}%
\subsubsection{Ecuaci'on de grafo planar}
$regiones = ejes - nodos + componentesConexas + 1$
\subsubsection{Ternas pitag'oricas}
Hay ternas pitag'oricas de la forma: $(a,b,c) = ( m^2-n^2 , 2\cdot m\cdot n, m^2+n^2 ) \forall m > n > 0 $\\
y son primitivas \emph{sii} $(2 | m\cdot n) \land (mcd(m, n) = 1)$\\
(Todas las primitivas (con $(a,b)$ no ordenado) son de esa forma.) Obs: $(\mathrm{m}+i\mathrm{n})^2 = \mathrm{a}+i\mathrm{b}$
\subsubsection{Teorema de Pick}
$A = I + \frac{B}{2} - 1$, donde $I =$ interior y $B =$ borde
%
\subsubsection{Propiedades varias}
$\sum_{i=0}^n{r^i} = \frac{r^{n+1}-1}{r-1}$ ; $\sum_{i=1}^n{i^2} = \frac{n\cdot(n+1)\cdot(2n+1)}{6}$ ;
$\sum_{i=1}^n{i^3} = \left(\frac{n\cdot(n+1)}{2}\right)^2$ \\
$\sum_{i=1}^n{i^4} = \frac{n\cdot(n+1)\cdot(2n+1)\cdot(3n^2+3n-1)}{12}$ ;
$\sum_{i=1}^{ n} i^5 = \left(\frac{ n \cdot (n+1)}{2}\right) ^2 \cdot \frac{2 n ^2 + 2n - 1}{3}$ \\
%
$\sum_{i=1}^n{\binom{n-1}{i-1}} = 2^{n-1}$ ; $\sum_{i=1}^n{i\cdot\binom{n-1}{i-1}} = n\cdot2^{n-1}$ \\
%
\subsubsection{Desarreglos}
$!n = n(!(n - 1)) + (-1)^n; n > 0$ 
\subsection{Tablas y cotas}
\subsubsection{Primos}
2 3 5 7 11 13 17 19 23 29
31 37 41 43 47 53 59 61 67 71
73 79 83 89 97 101 103 107 109 113
127 131 137 139 149 151 157 163 167 173
179 181 191 193 197 199 211 223 227 229
233 239 241 251 257 263 269 271 277 281
283 293 307 311 313 317 331 337 347 349
353 359 367 373 379 383 389 397 401 409
419 421 431 433 439 443 449 457 461 463
467 479 487 491 499 503 509 521 523 541
547 557 563 569 571 577 587 593 599 601
607 613 617 619 631 641 643 647 653 659
661 673 677 683 691 701 709 719 727 733
739 743 751 757 761 769 773 787 797 809
811 821 823 827 829 839 853 857 859 863
877 881 883 887 907 911 919 929 937 941
947 953 967 971 977 983 991 997 1009 1013
1019 1021 1031 1033 1039 1049 1051 1061 1063 1069
1087 1091 1093 1097 1103 1109 1117 1123 1129 1151
1153 1163 1171 1181 1187 1193 1201 1213 1217 1223
1229 1231 1237 1249 1259 1277 1279 1283 1289 1291
1297 1301 1303 1307 1319 1321 1327 1361 1367 1373
1381 1399 1409 1423 1427 1429 1433 1439 1447 1451
1453 1459 1471 1481 1483 1487 1489 1493 1499 1511
1523 1531 1543 1549 1553 1559 1567 1571 1579 1583
1597 1601 1607 1609 1613 1619 1621 1627 1637 1657
1663 1667 1669 1693 1697 1699 1709 1721 1723 1733
1741 1747 1753 1759 1777 1783 1787 1789 1801 1811
1823 1831 1847 1861 1867 1871 1873 1877 1879 1889
1901 1907 1913 1931 1933 1949 1951 1973 1979 1987
1993 1997 1999 2003 2011 2017 2027 2029 2039 2053
2063 2069 2081 2083 2087 2089 2099 2111 2113 2129
%2131 2137 2141 2143 2153 2161 2179 2203 2207 2213
%2221 2237 2239 2243 2251 2267 2269 2273 2281 2287
%2293 2297 2309 2311 2333 2339 2341 2347 2351 2357
%2371 2377 2381 2383 2389 2393 2399 2411 2417 2423
%2437 2441 2447 2459 2467 2473 2477 2503 2521 2531
%2539 2543 2549 2551 2557 2579 2591 2593 2609 2617
%2621 2633 2647 2657 2659 2663 2671 2677 2683 2687
%2689 2693 2699 2707 2711 2713 2719 2729 2731 2741
%2749 2753 2767 2777 2789 2791 2797 2801 2803 2819
%2833 2837 2843 2851 2857 2861 2879 2887 2897 2903
%2909 2917 2927 2939 2953 2957 2963 2969 2971 2999
%3001 3011 3019 3023 3037 3041 3049 3061 3067 3079
%3083 3089 3109 3119 3121 3137 3163 3167 3169 3181
%3187 3191 3203 3209 3217 3221 3229 3251 3253 3257
%3259 3271 3299 3301 3307 3313 3319 3323 3329 3331
%3343 3347 3359 3361 3371 3373 3389 3391 3407 3413
%3433 3449 3457 3461 3463 3467 3469 3491 3499 3511
%3517 3527 3529 3533 3539 3541 3547 3557 3559 3571\\
\paragraph{Primos cercanos a $10^n$}\ \\
9941 9949 9967 9973 10007 10009 10037 10039 10061 10067 10069 10079\\
99961 99971 99989 99991 100003 100019 100043 100049 100057 100069\\
999959 999961 999979 999983 1000003 1000033 1000037 1000039\\
9999943 9999971 9999973 9999991 10000019 10000079 10000103 10000121\\
99999941 99999959 99999971 99999989 100000007 100000037 100000039 100000049\\
999999893 999999929 999999937 1000000007 1000000009 1000000021 1000000033

\paragraph{Cantidad de primos menores que $10^n$}\ \\
$\pi(10^1)$ = 4 ;
$\pi(10^2)$ = 25 ;
$\pi(10^3)$ = 168 ;
$\pi(10^4)$ = 1229 ;
$\pi(10^5)$ = 9592 \\
$\pi(10^6)$ = 78.498 ;
$\pi(10^7)$ = 664.579 ;
$\pi(10^8)$ = 5.761.455 ;
$\pi(10^9)$ = 50.847.534 \\
$\pi(10^{10})$ = 455.052,511 ;
$\pi(10^{11})$ = 4.118.054.813 ;
$\pi(10^{12})$ = 37.607.912.018% ;
% Fuente: http://primes.utm.edu/howmany.shtml#table

\subsubsection{Divisores}
Cantidad de divisores ($\sigma_0$) para \emph{algunos} $n / \neg\exists n'<n, \sigma_0(n') \geqslant \sigma_0(n)$ \\
$\sigma_0(60)$ = 12 ; $\sigma_0(120)$ = 16 ; $\sigma_0(180)$ = 18 ; $\sigma_0(240)$ = 20 ; $\sigma_0(360)$ = 24 \\
$\sigma_0(720)$ = 30 ; $\sigma_0(840)$ = 32 ; $\sigma_0(1260)$ = 36 ; $\sigma_0(1680)$ = 40 ; $\sigma_0(10080)$ = 72 \\ $\sigma_0(15120)$ = 80 ; $\sigma_0(50400)$ = 108 ; $\sigma_0(83160)$ = 128 ; $\sigma_0(110880)$ = 144 \\
$\sigma_0(498960)$ = 200 ; $\sigma_0(554400)$ = 216 ; $\sigma_0(1081080)$ = 256 ; $\sigma_0(1441440)$ = 288 ; $\sigma_0(4324320)$ = 384 ; $\sigma_0(8648640)$ = 448 ;
$\sigma_0(1000000000)$ = 1344 ;

%
Suma de divisores ($\sigma_1$) para \emph{algunos} $n / \neg\exists n'<n, \sigma_1(n') \geqslant \sigma_1(n)$ \\
$\sigma_1(96)$ = 252 ; $\sigma_1(108)$ = 280 ; $\sigma_1(120)$ = 360 ; $\sigma_1(144)$ = 403 ; $\sigma_1(168)$ = 480 \\
$\sigma_1(960)$ = 3048 ; $\sigma_1(1008)$ = 3224 ; $\sigma_1(1080)$ = 3600 ; $\sigma_1(1200)$ = 3844 \\
$\sigma_1(4620)$ = 16128 ; $\sigma_1(4680)$ = 16380 ; $\sigma_1(5040)$ = 19344 ; $\sigma_1(5760)$ = 19890 \\
$\sigma_1(8820)$ = 31122 ; $\sigma_1(9240)$ = 34560 ; $\sigma_1(10080)$ = 39312 ; $\sigma_1(10920)$ = 40320 \\
$\sigma_1(32760)$ = 131040 ; $\sigma_1(35280)$ = 137826 ; $\sigma_1(36960)$ = 145152 ; $\sigma_1(37800)$ = 148800 \\
$\sigma_1(60480)$ = 243840 ; $\sigma_1(64680)$ = 246240 ; $\sigma_1(65520)$ = 270816 ; $\sigma_1(70560)$ = 280098 \\
$\sigma_1(95760)$ = 386880 ; $\sigma_1(98280)$ = 403200 ; $\sigma_1(100800)$ = 409448  \\
$\sigma_1(491400)$ = 2083200 ; $\sigma_1(498960)$ = 2160576 ; $\sigma_1(514080)$ = 2177280 \\
$\sigma_1(982800)$ = 4305280 ; $\sigma_1(997920)$ = 4390848 ; $\sigma_1(1048320)$ = 4464096 \\
$\sigma_1(4979520)$ = 22189440 ; $\sigma_1(4989600)$ = 22686048 ; $\sigma_1(5045040)$ = 23154768 \\
$\sigma_1(9896040)$ = 44323200 ; $\sigma_1(9959040)$ = 44553600 ; $\sigma_1(9979200)$ = 45732192
%
%
\subsubsection{Factoriales}
\begin{tabular}{l|l}
0! =	1             & 11! = 39.916.800  \\
1! =	1             & 12! =	479.001.600	($\in \mathtt{int}$)\\
2! =	2             & 13! =	6.227.020.800	\\
3! =	6             & 14! =	87.178.291.200	\\
4! =	24            & 15! =	1.307.674.368.000	\\
5! =	120   			  & 16! =	20.922.789.888.000	\\
6! =	720           & 17! =	355.687.428.096.000	\\
7! =	5.040	        & 18! =	6.402.373.705.728.000	\\
8! =	40.320	      & 19! =	121.645.100.408.832.000	\\
9! =	362.880       & 20! =	2.432.902.008.176.640.000	($\in \mathtt{tint}$) \\
10! =	3.628.800     & 21! =	51.090.942.171.709.400.000
\end{tabular}

max signed tint = 9.223.372.036.854.775.807 \\
max unsigned tint = 18.446.744.073.709.551.615
\subsection{FFT (Final Fourier Tactics)}
TODO: add operations
\begin{code}
// MAXN must be power of 2 !!
// MAXN = 131072, or MAXN = 1048576 
// remember that poly(n) * poly(m) = poly(n + m) to define MAXN
// MOD-1 needs to be a multiple of MAXN !!
// big mod and primitive root for NTT:
typedef ll tf;
typedef vector<tf> poly;
const tf MOD=2305843009255636993,RT=5;
const double pi = acos(-1.0)
// FFT
struct CD {
	double r,i;
	CD(double r=0, double i=0):r(r),i(i){}
	double real()const{return r;}
	void operator/=(const int c){r/=c, i/=c;}
};
CD operator*(const CD& a, const CD& b){
	return CD(a.r*b.r-a.i*b.i,a.r*b.i+a.i*b.r);}
CD operator+(const CD& a, const CD& b){return CD(a.r+b.r,a.i+b.i);}
CD operator-(const CD& a, const CD& b){return CD(a.r-b.r,a.i-b.i);}
const double pi=acos(-1.0);
// NTT
/*
struct CD {
	tf x;
	CD(tf x):x(x){}
	CD(){}
};
CD operator*(const CD& a, const CD& b){return CD(mulmod(a.x,b.x));}
CD operator+(const CD& a, const CD& b){return CD(addmod(a.x,b.x));}
CD operator-(const CD& a, const CD& b){return CD(submod(a.x,b.x));}
vector<tf> rts(MAXN+9,-1);
CD root(int n, bool inv){
	tf r=rts[n]<0?rts[n]=pm(RT,(MOD-1)/n):rts[n];
	return CD(inv?pm(r,MOD-2):r);
}
*/
CD cp1[MAXN+9],cp2[MAXN+9];
int R[MAXN+9];
void dft(CD* a, int n, bool inv){
	forn(i,n)if(R[i]<i)swap(a[R[i]],a[i]);
	for(int m=2;m<=n;m*=2){
		double z=2*pi/m*(inv?-1:1); // FFT
		CD wi=CD(cos(z),sin(z)); // FFT
		// CD wi=root(m,inv); // NTT
		for(int j=0;j<n;j+=m){
			CD w(1);
			for(int k=j,k2=j+m/2;k2<j+m;k++,k2++){
				CD u=a[k];CD v=a[k2]*w;a[k]=u+v;a[k2]=u-v;w=w*wi;
			}
		}
	}
	if(inv)forn(i,n)a[i]/=n; // FFT
	//if(inv){ // NTT
	//	CD z(pm(n,MOD-2)); // pm: modular exponentiation
	//	forsn(i,0,n)a[i]=a[i]*z;
	//}
}
poly multiply(poly& p1, poly& p2){
	int n=p1.size()+p2.size()+1;
	int m=1,cnt=0;
	while(m<=n)m+=m,cnt++;
	forn(i,m){R[i]=0;forn(j,cnt)R[i]=(R[i]<<1)|((i>>j)&1);}
	forn(i,m)cp1[i]=0,cp2[i]=0;
	forn(i,p1.size())cp1[i]=p1[i];
	forn(i,p2.size())cp2[i]=p2[i];
	dft(cp1,m,false);dft(cp2,m,false);
	forn(i,m)cp1[i]=cp1[i]*cp2[i];
	dft(cp1,m,true);
	poly res;
	n-=2;
	forn(i,n)res.pb((tf)floor(cp1[i].real()+0.5)); // FFT
	//forn(i,n)res.pb(cp1[i].x); // NTT
	return res;
}
\end{code}
\subsection{Berlekamp-Massey}
\begin{code}
vi BM(vi x){
	vi ls,cur;int lf,ld;
    forn(i,x.size()){
		ll t=0;
        forn(j,cur.size())t=(t+x[i-j-1]*(ll)cur[j])%MOD;
		if((t-x[i])%MOD==0)continue;
        if(!cur.size()){cur.resize(i+1);lf=i;ld=(t-x[i])%MOD;continue;}
		ll k=-(x[i]-t)*modPow(ld,MOD-2)%MOD;
		vi c(i-lf-1);c.pb(k);
        forn(j,ls.size())c.pb(-ls[j]*k%MOD);
        if(c.size()<cur.size())c.resize(cur.size());
        forn(j,cur.size())c[j]=(c[j]+cur[j])%MOD;
        if(i-lf+ls.size()>=cur.size())ls=cur,lf=i,ld=(t-x[i])%MOD;
		cur=c;
	}
    forn(i,cur.size())cur[i]=(cur[i]%MOD+MOD)%MOD;
	return cur;
}
\end{code}
\subsection{Pollard-Rho}
\begin{code}
ll gcd(ll a, ll b){return a?gcd(b%a,a):b;}
ll mulmod(ll a, ll b, ll m) {
	ll r=a*b-(ll)((long double)a*b/m+.5)*m;
	return r<0?r+m:r;
}
ll expmod(ll b, ll e, ll m){
	if(!e)return 1;
	ll q=expmod(b,e/2,m);q=mulmod(q,q,m);
	return e&1?mulmod(b,q,m):q;
}
bool is_prime_prob(ll n, int a){
	if(n==a)return true;
	ll s=0,d=n-1;
	while(d%2==0)s++,d/=2;
	ll x=expmod(a,d,n);
	if((x==1)||(x+1==n))return true;
	form(i,s-1){
		x=mulmod(x,x,n);
		if(x==1)return false;
		if(x+1==n)return true;
	}
	return false;
}
bool rabin(ll n){ // true iff n is prime
	if(n==1)return false;
	int ar[]={2,3,5,7,11,13,17,19,23};
	forn(i,9)if(!is_prime_prob(n,ar[i]))return false;
	return true;
}
ll rho(ll n){
	if(!(n&1))return 2;
	ll x=2,y=2,d=1;
	ll c=rand()%n+1;
	while(d==1){
		x=(mulmod(x,x,n)+c)%n;
		y=(mulmod(y,y,n)+c)%n;
		y=(mulmod(y,y,n)+c)%n;
		if(x>=y)d=gcd(x-y,n);
		else d=gcd(y-x,n);
	}
	return d==n?rho(n):d;
}
void fact(ll n, map<ll,int>& f){ //O (lg n)^3
	if(n==1)return;
	if(rabin(n)){f[n]++;return;}
	ll q=rho(n);fact(q,f);fact(n/q,f);
}
\end{code}
\subsection{Simplex}
\begin{code}
// TODO: define EPS
vector<int> X,Y;
vector<vector<double> > A;
vector<double> b,c;
double z;
int n,m;
void pivot(int x,int y){
	swap(X[y],Y[x]);
	b[x]/=A[x][y];
	forn(i,m)if(i!=y)A[x][i]/=A[x][y];
	A[x][y]=1/A[x][y];
	forn(i,n)if(i!=x&&abs(A[i][y])>EPS){
		b[i]-=A[i][y]*b[x];
		forn(j,m)if(j!=y)A[i][j]-=A[i][y]*A[x][j];
		A[i][y]=-A[i][y]*A[x][y];
	}
	z+=c[y]*b[x];
	forn(i,m)if(i!=y)c[i]-=c[y]*A[x][i];
	c[y]=-c[y]*A[x][y];
}
pair<double,vector<double> > simplex( // maximize c^T x s.t. Ax<=b, x>=0
		vector<vector<double> > _A, vector<double> _b, vector<double> _c){
	// returns pair (maximum value, solution vector)
	A=_A;b=_b;c=_c;
	n=b.size();m=c.size();z=0.;
	X=vector<int>(m);Y=vector<int>(n);
	forn(i,m)X[i]=i;
	forn(i,n)Y[i]=i+m;
	while(1){
		int x=-1,y=-1;
		double mn=-EPS;
		forn(i,n)if(b[i]<mn)mn=b[i],x=i;
		if(x<0)break;
		forn(i,m)if(A[x][i]<-EPS){y=i;break;}
		assert(y>=0); // no solution to Ax<=b
		pivot(x,y);
	}
	while(1){
		double mx=EPS;
		int x=-1,y=-1;
		forn(i,m)if(c[i]>mx)mx=c[i],y=i;
		if(y<0)break;
		double mn=1e200;
		forn(i,n)if(A[i][y]>EPS&&b[i]/A[i][y]<mn)mn=b[i]/A[i][y],x=i;
		assert(x>=0); // c^T x is unbounded
		pivot(x,y);
	}
	vector<double> r(m);
	forn(i,n)if(Y[i]<m)r[Y[i]]=b[i];
	return {z,r};
}
\end{code}
\section{Strings}
\subsection{Hash}
\begin{code}
struct Hash {
	int P=1777771, MOD[2], PI[2];
	vector<int> h[2],pi[2];
	Hash(string& s){
		MOD[0]=999727999; MOD[1]=1070777777;
		PI[0]=325255434; PI[1]=10018302;
		forn(k,2) h[k].resize(s.size()+1), pi[k].resize(s.size()+1);
		forn(k,2){
			h[k][0]=0;pi[k][0]=1;
			ll p=1;
			forsn(i,1,s.size()+1){
				h[k][i]=(h[k][i-1]+p*s[i-1])%MOD[k];
				pi[k][i]=(1LL*pi[k][i-1]*PI[k])%MOD[k];
				p=(p*P)%MOD[k];
			}
		}
	}
   	 // get hash of range [s, e] indexed from 0 to n - 1
	ll get(int s, int e){
	    	e++;
		ll h0=(h[0][e]-h[0][s]+MOD[0])%MOD[0];
		h0=(1LL*h0*pi[0][s])%MOD[0];
		ll h1=(h[1][e]-h[1][s]+MOD[1])%MOD[1];
		h1=(1LL*h1*pi[1][s])%MOD[1];
		return (h0<<32)|h1;
	}
};
//this hash is slower but accurate
#define bint __int128
struct Hash {
	bint MOD=212345678987654321LL,P=1777771,PI=106955741089659571LL;
	vector<bint> h,pi;
	Hash(string& s){
		assert((P*PI)%MOD==1);
		h.resize(s.size()+1);pi.resize(s.size()+1);
		h[0]=0;pi[0]=1;
		bint p=1;
		forsn(i,1,s.size()+1){
			h[i]=(h[i-1]+p*s[i-1])%MOD;
			pi[i]=(pi[i-1]*PI)%MOD;
			p=(p*P)%MOD;
		}
	}
	//get hash of range [s, e] indexed from 0 to n - 1
	ll get(int s, int e){
		e++;
		return (((h[e]-h[s]+MOD)%MOD)*pi[s])%MOD;
	}
};
\end{code}
\subsection{Suffix Array}
\begin{code}
#define RB(x) (x < n ? r[x] : 0)
 
void csort(vi& sa, vi& r, int k) {
    int n = sa.size();
    vi f(max(255, n), 0), t(n);
    forn(i, n) f[RB(i + k)]++;
    int sum  = 0;
    forn(i, max(255, n)) f[i] = (sum += f[i]) - f[i];
    forn(i, n) t[f[RB(sa[i] + k)]++] = sa[i];
    sa = t;
}
 
vi constructSA(string& s) { //O(nlogn)
    int n = s.size(), rank;
    vector<int> sa(n), r(n), t(n);
    forn(i, n) sa[i] = i, r[i] = s[i];
    for(int k = 1; k < n; k *= 2) {
        csort(sa, r, k); csort(sa, r, 0);
        t[sa[0]] = rank = 0;
        for(int i = 1; i < n; i++) {
            if(r[sa[i]] != r[sa[i - 1]] or RB(sa[i] + k) != RB(sa[i - 1] + k)) rank++;
            t[sa[i]] = rank;
        }
        r = t;
        if (r[sa[n - 1]] == n - 1) break;
    }
    return sa;
}

vector<int> computeLCP(string& s, vector<int>& sa){
	int n=s.size(),L=0;
	vector<int> lcp(n),plcp(n),phi(n);
	phi[sa[0]]=-1;
	forsn(i,1,n)phi[sa[i]]=sa[i-1];
	forn(i,n){
		if(phi[i]<0){plcp[i]=0;continue;}
		while(s[i+L]==s[phi[i]+L])L++;
		plcp[i]=L;
		L=max(L-1,0);
	}
	forn(i,n)lcp[i]=plcp[sa[i]];
	return lcp; // lcp[i]=LCP(sa[i-1],sa[i])
}
 
int patternMatching(string& s1, string& s2, vi& sa) { //O(s1log(s2))
    if (s2.size() < s1.size()) {
        return 0;
    }
    int result_start = 0, result_end = s2.size() - 1;
    forn(i, s1.size()) {
        int start = lower_bound(sa.begin() + result_start, sa.begin() + result_end + 1, s1[i], [&](int idx, char b) {
            return idx + i >= s2.size() || s2[idx + i] < b;
        }) - sa.begin();
        if (start > result_end || (s2[sa[start] + i]) != s1[i]) {
            return 0;
        }
        int end = upper_bound(sa.begin() + result_start, sa.begin() + result_end + 1, s1[i], [&](char b, int idx) {
            return idx + i < s2.size() && s2[idx + i] > b;
        }) - sa.begin() - 1;
        result_start = start;
        result_end = end;
    }
    return result_end - result_start + 1;
}
\end{code}
\subsection{Aho-Corasick}
\begin{code}
struct vertex {
    map<char,int> next,go;
    int p,link;
    char pch;
    vector<int> leaf;
    vertex(int p=-1, char pch=-1):p(p),pch(pch),link(-1){}
};
vector<vertex> t;
void aho_init(){ //do not forget!!
    t.clear();t.pb(vertex());
}
void add_string(string s, int id){
    int v=0;
    for(char c:s){
        if(!t[v].next.count(c)){
            t[v].next[c]=t.size();
            t.pb(vertex(v,c));
        }
        v=t[v].next[c];
    }
    t[v].leaf.pb(id);
}
int go(int v, char c);
int get_link(int v){
    if(t[v].link<0)
        if(!v||!t[v].p)t[v].link=0;
        else t[v].link=go(get_link(t[v].p),t[v].pch);
    return t[v].link;
}
int go(int v, char c){
    if(!t[v].go.count(c))
        if(t[v].next.count(c))t[v].go[c]=t[v].next[c];
        else t[v].go[c]=v==0?0:go(get_link(v),c);
    return t[v].go[c];
}
\end{code}
\subsection{Manacher's Algorithm}
\begin{code}
int d1[1000000];//d1[i] = max odd palindrome centered on i
int d2[1000000];//d2[i] = max even palindrome centered on i
//s  aabbaacaabbaa
//d1 1111117111111
//d2 0103010010301
void manacher(string& s){
    int l=0,r=-1,n=s.size();
    forn(i,n){
        int k=i>r?1:min(d1[l+r-i],r-i);
        while(i+k<n&&i-k>=0&&s[i+k]==s[i-k])k++;
        d1[i]=k--;
        if(i+k>r)l=i-k,r=i+k;
    }
    l=0;r=-1;
    forn(i,n){
        int k=i>r?0:min(d2[l+r-i+1],r-i+1);k++;
        while(i+k<=n&&i-k>=0&&s[i+k-1]==s[i-k])k++;
        d2[i]=--k;
        if(i+k-1>r)l=i-k,r=i+k-1;
    }
}
\end{code}
\section{Flujos}
\subsection{Dinic? No era Dinitz??}
Runs in $O(EV^2)$
\begin{code}
// to check if edge is saturated cap == 0
struct edge { int v, cap, inv, flow; };
struct network {
  int n, s, t;
  vector<int> lvl;
  vector<vector<edge>> g;
  network(int n) : n(n), lvl(n), g(n) {}
  void add_edge(int u, int v, int c) {
    g[u].push_back({v, c, g[v].size(), 0});
    g[v].push_back({u, 0, g[u].size()-1,c});
    // The following line is for undirected graphs
    // g[v].push_back({u, c, g[u].size()-1, 0});
  }
  bool bfs() {
    fill(lvl.begin(), lvl.end(), -1);
    queue<int> q;
    lvl[s] = 0;
    for(q.push(s); q.size(); q.pop()) {
      int u = q.front();
      for(auto &e : g[u]) {
        if(e.cap > 0 && lvl[e.v] == -1) {
          lvl[e.v] = lvl[u]+1;
          q.push(e.v);
        }
      }
    }
    return lvl[t] != -1;
  }
  int dfs(int u, int nf) {
    if(u == t) return nf;
    int res = 0;
    for(auto &e : g[u]) {
      if(e.cap > 0 && lvl[e.v] == lvl[u]+1) {
        int tf = dfs(e.v, min(nf, e.cap));
        res += tf; nf -= tf; e.cap -= tf;
        g[e.v][e.inv].cap += tf;
        g[e.v][e.inv].flow -= tf;
        e.flow += tf;
        if(nf == 0) return res;
      }
    }
    if(!res) lvl[u] = -1;
    return res;
  }
  int max_flow(int so, int si, int res = 0) {
    s = so; t = si;
    while(bfs()) res += dfs(s, INT_MAX);
    return res;
  }
  //first find max flow
  vector<ii> minCut() {
    bfs();
    vector<ii> mc;
    forn(i, n) {
        if(lvl[i] == -1) continue;
        for(auto &e : g[i]) {
            if(lvl[e.v] == -1) {
                mc.pb({i, e.v});
            }
        }
    }
    return mc;
  }
};
\end{code}
\subsection{Hungarian Matching}
Runs in $O(V^3)$ \\
Algorithm for Min cost max flow in bipartite graph
\begin{code}
typedef int type;
struct matching_weighted {
  int l, r; //n m
  vector<vector<type>> c;
  matching_weighted(int l, int r) : l(l), r(r), c(l, vector<type>(r)) {
    assert(l <= r);
  }
  void add_edge(int a, int b, type cost) { c[a][b] = cost; }
  type matching() {
    vector<type> v(r), d(r); // v: potential
    vector<int> ml(l, -1), mr(r, -1); // matching pairs
    vector<int> idx(r), prev(r);
    iota(idx.begin(), idx.end(), 0);
    auto residue = [&](int i, int j) { return c[i][j]-v[j]; };
    for(int f = 0; f < l; ++f) {
      for(int j = 0; j < r; ++j) {
        d[j] = residue(f, j);
        prev[j] = f;
      }
      type w;
      int j, l;
      for (int s = 0, t = 0;;) {
        if(s == t) {
          l = s;
          w = d[idx[t++]];
          for(int k = t; k < r; ++k) {
            j = idx[k];
            type h = d[j];
            if (h <= w) {
              if (h < w) t = s, w = h;
              idx[k] = idx[t];
              idx[t++] = j;
            }
          }
          for (int k = s; k < t; ++k) {
            j = idx[k];
            if (mr[j] < 0) goto aug;
          }
        }
        int q = idx[s++], i = mr[q];
        for (int k = t; k < r; ++k) {
          j = idx[k];
          type h = residue(i, j) - residue(i, q) + w;
          if (h < d[j]) {
            d[j] = h;
            prev[j] = i;
            if(h == w) {
              if(mr[j] < 0) goto aug;
              idx[k] = idx[t];
              idx[t++] = j;
            }
          }
        }
      }
      aug: for (int k = 0; k < l; ++k)
        v[ idx[k] ] += d[ idx[k] ] - w;
      int i;
      do {
        mr[j] = i = prev[j];
        swap(j, ml[i]);
      } while (i != f);
    }
    type opt = 0;
    for (int i = 0; i < l; ++i)
      opt += c[i][ml[i]]; // (i, ml[i]) is a solution
    return opt;
  }
};
\end{code}
\subsection{mcmf}
Runs in $O(VE + F(ElogV)$ where F is the max flow
\begin{code}
typedef ll tf;
typedef ll tc;
const tf INFFLOW=1e9;
const tc INFCOST=1e9;
struct MCF{
	int n;
	vector<tc> prio, pot; vector<tf> curflow; vector<int> prevedge,prevnode;
	priority_queue<pair<tc, int>, vector<pair<tc, int>>, greater<pair<tc, int>>> q;
	struct edge{int to, rev; tf f, cap; tc cost;};
	vector<vector<edge>> g;
	MCF(int n):n(n),prio(n),curflow(n),prevedge(n),prevnode(n),pot(n),g(n){}
	void add_edge(int s, int t, tf cap, tc cost) {
		g[s].pb((edge){t,SZ(g[t]),0,cap,cost});
		g[t].pb((edge){s,SZ(g[s])-1,0,0,-cost});
	}
	pair<tf,tc> get_flow(int s, int t) {
		tf flow=0; tc flowcost=0;
		while(1){
			q.push({0, s});
			fill(ALL(prio),INFCOST); 
			prio[s]=0; curflow[s]=INFFLOW;
			while(!q.empty()) {
				auto cur=q.top();
				tc d=cur.F;
				int u=cur.S;
				q.pop();
				if(d!=prio[u]) continue;
				for(int i=0; i<SZ(g[u]); ++i) {
					edge &e=g[u][i];
					int v=e.to;
					if(e.cap<=e.f) continue;
					tc nprio=prio[u]+e.cost+pot[u]-pot[v];
					if(prio[v]>nprio) {
						prio[v]=nprio;
						q.push({nprio, v});
						prevnode[v]=u; prevedge[v]=i;
						curflow[v]=min(curflow[u], e.cap-e.f);
					}
				}
			}
			if(prio[t]==INFCOST) break;
			forn(i,n) pot[i]+=prio[i];
			tf df=min(curflow[t], INFFLOW-flow);
			flow+=df;
			for(int v=t; v!=s; v=prevnode[v]) {
				edge &e=g[prevnode[v]][prevedge[v]];
				e.f+=df; g[v][e.rev].f-=df;
				flowcost+=df*e.cost;
			}
		}
		return {flow,flowcost};
	}
};
\end{code}
\subsection{Hopcroft Karp Karzanov (HKK)}
Max matching in bipartite graph
Runs in $O(E\sqrt{V})$
\begin{code}
vector<int> g[MAXN]; // [0,n)->[0,m)
int n,m;
int mt[MAXN],mt2[MAXN],ds[MAXN];
bool bfs(){
	queue<int> q;
	memset(ds,-1,sizeof(ds));
	forn(i,n)if(mt2[i]<0)ds[i]=0,q.push(i);
	bool r=false;
	while(!q.empty()){
		int x=q.front();q.pop();
		for(int y:g[x]){
			if(mt[y]>=0&&ds[mt[y]]<0)ds[mt[y]]=ds[x]+1,q.push(mt[y]);
			else if(mt[y]<0)r=true;
		}
	}
	return r;
}
bool dfs(int x){
	for(int y:g[x])if(mt[y]<0||ds[mt[y]]==ds[x]+1&&dfs(mt[y])){
		mt[y]=x;mt2[x]=y;
		return true;
	}
	ds[x]=1<<30;
	return false;
}
int mm(){
	int r=0;
	memset(mt,-1,sizeof(mt));memset(mt2,-1,sizeof(mt2));
	while(bfs()){
		forn(i,n)if(mt2[i]<0)r+=dfs(i);
	}
	return r;
}
\end{code}
\section{Locuras}
\subsection{Euler Circuit}
Runs in $O(E)$\\
Check si hay circuito euleriano primero, si hay camino agregar eje entre nodos de grado impar \\
\begin{code}
//the graph shoud store edge as the second parameter of a pair
bool changeStart = false;
int start = 0;
deque<int> eulerCircuit;
bool usedEdges[200005];
 
void changeCircuitStart(int s) {
    eulerCircuit.pop_back();
    while(eulerCircuit.back() != s) {
        eulerCircuit.pf(eulerCircuit.back());
        eulerCircuit.pop_back();
    }
    eulerCircuit.pf(s);
}
//in case we had to add an edge between odd degree vertices
void circuitToPath(int e) {
    eulerCircuit.pop_back();
    while(eulerCircuit.back() != e) {
        eulerCircuit.pf(eulerCircuit.back());
        eulerCircuit.pop_back();
    }
}
void findCircuit(int u) {
    for(auto p : g[u]) {
        int v = p.F;
        int id = p.S;
        if(!usedEdges[id]) {
            if(changeStart) {
                changeStart = false;
                changeCircuitStart(u); start = u;
            }
            usedEdges[id] = true;
            eulerCircuit.pb(v);
            if(v == start) {
                if(u != v)  {
                    changeStart = true;
                }
            }
            else {
                findCircuit(v);   
            }
        }
    }
    // circuit is at eulerCircuit
    // for euler path changePath to start then use circuitToPath with end
}
\end{code}
\subsection{Chu Liu}
Find MST on directed graph
Runs in $O(nm)$
\begin{code}
//returns -1 if not possible
//included i-th edge if take[i]!=0
typedef int tw; tw INF=1ll<<30;
struct edge{int u,v,id;tw len;};
struct ChuLiu{
	int n; vector<edge> e;
	vector<int> inc,dec,take,pre,num,id,vis;
	vector<tw> inw;
	void add_edge(int x, int y, tw w){
		inc.pb(0); dec.pb(0); take.pb(0);
		e.pb({x,y,SZ(e),w});
	}
	ChuLiu(int n):n(n),pre(n),num(n),id(n),vis(n),inw(n){}
	tw doit(int root){
		auto e2=e;
		tw ans=0; int eg=SZ(e)-1,pos=SZ(e)-1;
		while(1){
			forn(i,n) inw[i]=INF,id[i]=vis[i]=-1;
			for(auto ed:e2) if(ed.len<inw[ed.v]){
				inw[ed.v]=ed.len; pre[ed.v]=ed.u;
				num[ed.v]=ed.id;
			}
			inw[root]=0;
			forn(i,n) if(inw[i]==INF) return -1;
			int tot=-1;
			forn(i,n){
				ans+=inw[i];
				if(i!=root)take[num[i]]++;
				int j=i;
				while(vis[j]!=i&&j!=root&&id[j]<0)vis[j]=i,j=pre[j];
				if(j!=root&&id[j]<0){
					id[j]=++tot;
					for(int k=pre[j];k!=j;k=pre[k]) id[k]=tot;
				}
			}
			if(tot<0)break;
			forn(i,n) if(id[i]<0)id[i]=++tot;
			n=tot+1; int j=0;
			forn(i,SZ(e2)){
				int v=e2[i].v;
				e2[j].v=id[e2[i].v];
				e2[j].u=id[e2[i].u];
				if(e2[j].v!=e2[j].u){
					e2[j].len=e2[i].len-inw[v];
					inc.pb(e2[i].id);
					dec.pb(num[v]);
					take.pb(0);
					e2[j++].id=++pos;
				}
			}
			e2.resize(j);
			root=id[root];
		}
		while(pos>eg){
			if(take[pos]>0) take[inc[pos]]++, take[dec[pos]]--;
			pos--;
		}
		return ans;
	}
};
\end{code}
\subsection{Dynamic Connectivity}
runs ofline
\begin{code}
struct UnionFind {
	int n,comp;
	vector<int> uf,si,c;
	UnionFind(int n=0):n(n),comp(n),uf(n),si(n,1){
		forn(i,n)uf[i]=i;}
	int find(int x){return x==uf[x]?x:find(uf[x]);}
	bool join(int x, int y){
		if((x=find(x))==(y=find(y)))return false;
		if(si[x]<si[y])swap(x,y);
		si[x]+=si[y];uf[y]=x;comp--;c.pb(y);
		return true;
	}
	int snap(){return c.size();}
	void rollback(int snap){
		while(c.size()>snap){
			int x=c.back();c.pop_back();
			si[uf[x]]-=si[x];uf[x]=x;comp++;
		}
	}
};
enum {ADD,DEL,QUERY};
struct Query {int type,x,y;};
struct DynCon {
	vector<Query> q;
	UnionFind dsu;
	vector<int> mt;
	map<pair<int,int>,int> last;
	DynCon(int n):dsu(n){}
	void add(int x, int y){
		if(x>y)swap(x,y);
		q.pb((Query){ADD,x,y});mt.pb(-1);last[{x,y}]=q.size()-1;
	}
	void remove(int x, int y){
		if(x>y)swap(x,y);
		q.pb((Query){DEL,x,y});
		int pr=last[{x,y}];mt[pr]=q.size()-1;mt.pb(pr);
	}
	void query(){q.pb((Query){QUERY,-1,-1});mt.pb(-1);}
	void process(){ // answers all queries in order
		if(!q.size())return;
		forn(i,q.size())if(q[i].type==ADD&&mt[i]<0)mt[i]=q.size();
		go(0,q.size());
	}
	void go(int s, int e){
		if(s+1==e){
			if(q[s].type==QUERY) // answer query using DSU
				printf("%d\n",dsu.comp);
			return;
		}
		int k=dsu.snap(),m=(s+e)/2;
		for(int i=e-1;i>=m;--i)if(mt[i]>=0&&mt[i]<s)dsu.join(q[i].x,q[i].y);
		go(s,m);dsu.rollback(k);
		for(int i=m-1;i>=s;--i)if(mt[i]>=e)dsu.join(q[i].x,q[i].y);
		go(m,e);dsu.rollback(k);
	}
};
\end{code}
\subsection{SCC y 2sat}
\begin{code}
// MAXN: max number of nodes or 2 * max number of variables (2SAT)
bool truth[MAXN]; // truth[cmp[i]]=value of variable i (2SAT)
int nvar;int neg(int x){return MAXN-1-x;} // (2SAT)
vector<int> g[MAXN];
int n,lw[MAXN],idx[MAXN],qidx,cmp[MAXN],qcmp;
stack<int> st;
void tjn(int u){
	lw[u]=idx[u]=++qidx;
	st.push(u);cmp[u]=-2;
	for(int v:g[u]){
		if(!idx[v]||cmp[v]==-2){
			if(!idx[v]) tjn(v);
			lw[u]=min(lw[u],lw[v]);
		}
	}
	if(lw[u]==idx[u]){
		int x,l=-1;
		do{x=st.top();st.pop();cmp[x]=qcmp;if(min(x,neg(x))<nvar)l=x;}
		while(x!=u);
		if(l!=-1)truth[qcmp]=(cmp[neg(l)]<0); // (2SAT)
		qcmp++;
	}
}
void scc(){
	memset(idx,0,sizeof(idx));qidx=0;
	memset(cmp,-1,sizeof(cmp));qcmp=0;
	forn(i,n)if(!idx[i])tjn(i);
}
// Only for 2SAT:
void addor(int a, int b){g[neg(a)].pb(b);g[neg(b)].pb(a);}
bool satisf(int _nvar){
	nvar=_nvar;n=MAXN;scc();
	forn(i,nvar)if(cmp[i]==cmp[neg(i)])return false;
	return true;
}
\end{code}
\section{No me gustaban los que eran de Geometr\'ia}
\subsection{Cross and Dot product}
\begin{code}
//returns orientation with respect to 0; 
//< 0 -> b is to the left of a
//> 0 -> b is to the right of a
//==0 -> b is colinear with a
//also returns area of paralelogram form by vectors
ll cross(ii a, ii b) {
    return a.F * b.S - a.S * b.F;
}
ll dot(ii a, ii b) {
    return a.F * b.F + a.S * b.S;
}
\end{code}
\subsection{Point in Poly}
\begin{code}
//TODO implement cross product
bool pointInLine(ii a, ii s, ii ed) {
    ii l = ii(s.F - a.F, s.S - a.S);
    ii r = ii(ed.F - a.F, ed.S - a.S);
    if(cross(l, r) == 0 and min(s.F, ed.F) <= a.F and a.F <= max(s.F, ed.F)
        and min(s.S, ed.S) <= a.S and a.S <= max(s.S, ed.S)) 
        return true;
    return false;
}
//0 -> outside, 1 -> inside, 2 -> boundary
int insidePoly(vector<ii> poly, ii point) {
    poly.pb(poly[0]);
    int cont = 0;
    forn(i, poly.size() - 1) {
        if(pointInLine(point, poly[i], poly[i + 1])) return 2;
        if(poly[i].F <= point.F and point.F < poly[i + 1].F) {
            ii l = ii(poly[i].F - point.F, poly[i].S - point.S);
            ii r = ii(poly[i + 1].F - point.F, poly[i + 1].S - point.S);
            if(cross(l, r) < 0) cont++;
        }
        else if(poly[i + 1].F <= point.F and point.F < poly[i].F) {
            ii r = ii(poly[i].F - point.F, poly[i].S - point.S);
            ii l = ii(poly[i + 1].F - point.F, poly[i + 1].S - point.S);
            if(cross(l, r) < 0) cont++;
        }
    }
    if(cont%2 == 0) return 0;
    return 1;
}
\end{code}
\subsection{Min Euclid Dist}
\begin{code}
ll distSqrd(ii a, ii b) {
    return (ll)((a.F - b.F) * (ll)(a.F - b.F)) + (ll)((a.S - b.S) * (ll)(a.S - b.S));
} 
//return dist * dist, get sqrt for real dist
ll minEuclidDist(vector<ii> p) {
    sort(all(p));
    set<ii> s;
    ll minDistSqr = 8000000000000000001;
    ll minDistP1 = sqrt(minDistSqr) + 1;
    int l = 0;
    forn(i, p.size()) {
        auto it = s.lower_bound(mp((ll)p[i].S - minDistP1, 0LL));
        while(it != s.end() and it->F <= p[i].S + minDistP1) {
            ii pe = (*it);
            swap(pe.F, pe.S);
            ll nd = distSqrd(p[i], pe);
            if(nd < minDistSqr) {
                minDistSqr = nd;
                minDistP1 = sqrt(minDistSqr);
            }
            it++;
        }
        s.insert(mp(p[i].S, p[i].F));
        while(l < i) {
            if(abso(p[i].F - p[l].F) > minDistP1) {
                s.erase(mp(p[l].S, p[l].F));
                l++;
            }
            else break;
        }
    }
    return minDistSqr;
}
\end{code}
\subsection{Segment intersection}
\begin{code}
ii toVect(ii a, ii b) {
    return{b.F - a.F, b.S - a.S};
} 
bool segmentIntersection(ii a, ii b, ii c, ii d){
    ll v1 = cross(toVect(c, a), toVect(c, b));
    ll v2 = cross(toVect(d, a), toVect(d, b));
    if((v1 < 0 and v2 < 0) or (v1 > 0 and v2 > 0)) {
        return false;
    }
    if(v1 == 0 and v2 == 0) {
        if((min(a.F, b.F) <= c.F and c.F <= max(a.F, b.F)) and
            (min(a.S, b.S) <= c.S and c.S <= max(a.S, b.S))) {
                return true;
            }
        if((min(a.x, b.x) <= d.x and d.x <= max(a.x, b.x)) and
            (min(a.y, b.y) <= d.y and d.y <= max(a.y, b.y))) {
                return true;
            }
        if((min(c.x, d.x) <= a.x and a.x <= max(c.x, d.x)) and
            (min(c.y, d.y) <= a.y and a.y <= max(c.y, d.y))) {
                return true;
            }
        if((min(c.x, d.x) <= b.x and b.x <= max(c.x, d.x)) and
            (min(c.y, d.y) <= b.y and b.y <= max(c.y, d.y))) {
                return true;
            }
        return false;
    }
    
    v1 = cross(toVect(a, c), toVect(a, d));
    v2 = cross(toVect(b, c), toVect(b, d));
    
    if((v1 < 0 and v2 < 0) or (v1 > 0 and v2 > 0)) {
        return false;
    }
    return true;
}
\end{code}
\subsection{Radial Sort}
\begin{code}
struct pt {  // for 3D add z coordinate
	double x,y;
	pt(double x, double y):x(x),y(y){}
	pt(){}
	double norm2(){return *this**this;}
	double norm(){return sqrt(norm2());}
	bool operator==(pt p){return abs(x-p.x)<=EPS&&abs(y-p.y)<=EPS;}
	pt operator+(pt p){return pt(x+p.x,y+p.y);}
	pt operator-(pt p){return pt(x-p.x,y-p.y);}
	pt operator*(double t){return pt(x*t,y*t);}
	pt operator/(double t){return pt(x/t,y/t);}
	double operator*(pt p){return x*p.x+y*p.y;}
//	pt operator^(pt p){ // only for 3D
//		return pt(y*p.z-z*p.y,z*p.x-x*p.z,x*p.y-y*p.x);}
	double angle(pt p){ // redefine acos for values out of range
		return acos(*this*p/(norm()*p.norm()));}
	pt unit(){return *this/norm();}
	double operator%(pt p){return x*p.y-y*p.x;}
	// 2D from now on
	bool operator<(pt p)const{ // for convex hull
		return x<p.x-EPS||(abs(x-p.x)<=EPS&&y<p.y-EPS);}
	bool left(pt p, pt q){ // is it to the left of directed line pq?
		return (q-p)%(*this-p)>EPS;}
	pt rot(pt r){return pt(*this%r,*this*r);}
	pt rot(double a){return rot(pt(sin(a),cos(a)));}
};
pt ccw90(1,0);
pt cw90(-1,0);
struct Cmp { // IMPORTANT: add const in pt operator -
	pt r;
	Cmp(pt r):r(r){}
	int cuad(const pt &a)const {
		if(a.x>0&&a.y>=0)return 0;
		if(a.x<=0&&a.y>0)return 1;
		if(a.x<0&&a.y<=0)return 2;
		if(a.x>=0&&a.y<0)return 3;
		assert(a.x==0&&a.y==0);
		return -1;
	}
	bool cmp(const pt& p1, const pt& p2)const {
		int c1=cuad(p1),c2=cuad(p2);
		if(c1==c2)return p1.y*p2.x<p1.x*p2.y;
		return c1<c2;
	}
	bool operator()(const pt& p1, const pt& p2)const {
		return cmp(p1-r,p2-r);
	}
};
\end{code}
\section{Estaba entre el Greedy y la DP}
\subsection{Chull Trick Dynamic}
Find max/min value on set of lines, allows dynamic insertion.\\
This implementation finds max, for min insert $-mx - b$, of course the result of eval will be $-res$\\
Works for dp of the form $mx + b$
\begin{code}
typedef ll tc;
const tc is_query=-(1LL<<62); // special value for query
struct Line {
	tc m,b;
	mutable multiset<Line>::iterator it,end;
	const Line* succ(multiset<Line>::iterator it) const {
		return (++it==end? NULL : &*it);}
	bool operator<(const Line& rhs) const {
		if(rhs.b!=is_query)return m<rhs.m;
		const Line *s=succ(it);
		if(!s)return 0;
		return b-s->b<(s->m-m)*rhs.m;
	}
};
struct CHT : public multiset<Line> { // for maximum
	bool bad(iterator y){
		iterator z=next(y);
		if(y==begin()){
			if(z==end())return false;
			return y->m==z->m&&y->b<=z->b;
		}
		iterator x=prev(y);
		if(z==end())return y->m==x->m&&y->b<=x->b;
		return (x->b-y->b)*(z->m-y->m)>=(y->b-z->b)*(y->m-x->m);
	}
	iterator next(iterator y){return ++y;}
	iterator prev(iterator y){return --y;}
	void add(tc m, tc b){
		iterator y=insert((Line){m,b});
		y->it=y;y->end=end();
		if(bad(y)){erase(y);return;}
		while(next(y)!=end()&&bad(next(y)))erase(next(y));
		while(y!=begin()&&bad(prev(y)))erase(prev(y));
	}
	tc eval(tc x){
		Line l=*lower_bound((Line){x,is_query});
		return l.m*x+l.b;
	}
};
\end{code}
\section{Detalles}
\subsection{int 128}
\begin{code}
__int128 read128(){
	__int128 x = 0, f = 1;
	char ch = getchar();
	while(ch < '0' or ch > '9'){
		if(ch == '-') f = -1;
		ch = getchar();
	}
	while(ch >= '0' && ch <= '9'){
		x = x * 10 + ch - '0';
		ch = getchar();
	}
	return x * f;
}

void print(__int128 x){
	if(x < 0){
		putchar('-');
		x = -x;
	}
	if(x > 9) print(x/10);
	putchar(x%10 + '0');
}

bool cmp(__int128 x, __int128 y){
	return x > y;
}
\end{code}
\subsection{Evitar colisiones en unordered\_map/set}
\begin{code}
struct custom_hash {
    static uint64_t splitmix64(uint64_t x) {
        x += 0x9e3779b97f4a7c15;
        x = (x ^ (x >> 30)) * 0xbf58476d1ce4e5b9;
        x = (x ^ (x >> 27)) * 0x94d049bb133111eb;
        return x ^ (x >> 31);
    }
 
    size_t operator()(uint64_t x) const {
        static const uint64_t FIXED_RANDOM = chrono::steady_clock::now().time_since_epoch().count();
        return splitmix64(x + FIXED_RANDOM);
    }
};
//unordered_map<int, int, custom_hash> m
\end{code}
\subsection{Matrix}
\begin{code}
class Matrix{
public:
    int r, c;
    vector<vi> m;
    Matrix(int rows, int cols) : r(rows), c(cols){
        m = vector<vi>(r, vi(c, 0));
    }
    Matrix(int n) : Matrix(n, n){
        forn(i, r) m[i][i] = 1; 
    }
    Matrix(vector<vi> mt) : Matrix(mt.size(), mt[0].size()) {
        m = mt;
    }
    Matrix operator * (Matrix b) {
        Matrix res(r, b.c);
        forn(i, r) {
            forn(j, b.c) {
                forn(k, c) {
                    res.m[i][j] = (res.m[i][j] + (m[i][k] * b.m[k][j])%M)%M;
                }
            }
        }
        return res;
    }
};
\end{code}
\newpage
\section{El Himno nacional}
Me encontraba en dif\'icil situaci\'on\\
En una prueba muy dif\'icil de mi vida \\
Estaba entre el Greedy y la DP \\
Mirando el F que no me sal\'ia  \\
Yo no queria una prueba normal \\
No me gustaban los que eran de geometr\'ia \\
El c\'odigo del C no me servia \\
Para dos rectas encontrar su intersecci\'on \\
Y tuve una revelaci\'on \\
Se calcular la geometr\'ia \\
Voy a evitar el overflow \\
\textit{Tristemente, este verso se perdio en el tiempo} \\
\\
Porque yo \\
No quiero debuggear \\
No quiero ya testear \\
No quiero submitear \\
Quiero ganarme el accepted enseguida \\
\textit{Tristemente, este verso se perdio en el tiempo} \\
\\
En la cabeza tenia arapos de mi viejo \\
Que resonaban como rulo de tambor \\
Vos, mejor que debugges \\
Mejor que programes \\
Y que obtengase 10 \\
\\
Ya me canse de que no metas los problemas \\
Voy a codearme la DP de las monedas \\
Vos mejor que debugges \\
Mejor que programes \\
Y que obtengase 10 \\
Ya me canse de ser tu fuente de algoritmos \\
\textit{El resto de la canci\'on se perdio en el tiempo} \\
\\
\\
\textit{Gracias TCA por todo lo que me enseñaste estos años} \\

\newpage
\begin{verbatim}
Baby, baby, don't look so sad
There's gonna be a better tomorrow
重い扉の向こうは
いつでも 青空さ
昨日と同じ一日が暮れて
彼女は深い溜息とともに眠る
果たせなかった約束
またひとつ増えただけ
それでも明日を夢見る
Baby, baby, close your eyes
Go back into your endless dream
目覚める頃はとっくに
笑顔が戻ってる
いい事だけを信じてるうちに
すべてを許せる自分に会える
いつか
あんなに愛した人も
愛してくれた人も
振り向けば ただの幻
Baby, baby, don't think you're lonely
Don't give up loving somebody new
繰り返される別れに
臆病にならないで
Ah, 淋しいなんて
Ah, 感じるひまもないくらい
Uh-uh-uh, uh-uh-uh, uh-uh
Uh-uh-uh, uh-uh-uh, uh-uh
果たせなかった約束
またひとつ増えただけ
それでも明日を夢見る
Baby, baby, don't look so sad
There's gonna be a better tomorrow
重い扉の向こうは
いつでも 青空さ
Baby, baby, close your eyes
Go back into your endless dream
果てしない夢の続き
見させてあげるから

竹内まりやの「夢の続き」
\end{verbatim}
\end{document}
